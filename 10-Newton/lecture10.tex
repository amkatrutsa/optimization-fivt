\documentclass{beamer}
\mode<presentation>
{\usetheme{default}
 \usecolortheme{default}
 \usefonttheme{default}
 \setbeamertemplate{navigation symbols}{}
 \setbeamertemplate{caption}[numbered]} 
\usepackage[russian]{babel}
\usepackage[utf8]{inputenc}
\usepackage[T2A]{fontenc}
\usepackage{graphicx}
\usepackage{subcaption}
\newtheorem{Th}{Теорема}
%%%%%%%%%%%%%%%%%%%%%%%%%%%%%%%%%%%%%%%%%%%%%%
% Formatting for title page
\addtobeamertemplate{navigation symbols}{}{    \usebeamerfont{footline}%
    \insertframenumber \ /  \inserttotalframenumber }
    
    
\title[Лекция 10]{Методы оптимизации\\ Лекция 10: Метод Ньютона. Квазиньютоновские методы}
\author{Александр Катруца}
\institute{
Факультет инноваций и высоких технологий \\ 
Физтех-школа прикладной математики и информатики\\
\vspace{-0.5cm}
\begin{figure}
\begin{subfigure}[c]{0.3\textwidth}
\includegraphics[scale=0.3]{../pics/fivt_logo}
\end{subfigure}
~
\begin{subfigure}[c]{0.3\textwidth}
 \includegraphics[scale=0.05]{../pics/fpmi_logo}
\end{subfigure}
\end{figure}
\vspace{-1cm}
}
\date{\today}
%%%%%%%%%%%%%%%%%%%%%%%%%%%%%%%%%%%%%%%%%%%%%%
\begin{document}
\begin{frame}
  \titlepage
\end{frame}

\begin{frame}{Метод Ньютона}
\[
x_{k+1} = x_k - f''(x_k)^{-1}f'(x_k)
\]
\end{frame}

\begin{frame}{Скорость сходимости-1}

\end{frame}

\begin{frame}{Скорость сходимости-2}

\begin{Th}
Пусть 
\begin{itemize}
\item<1->  $f(x)$ локально сильно выпукла с константой $\mu$: $\exists \ x^*: \; f''(x^*) \succeq \mu I$
\item<2-> гессиан Липшицев: $\| f''(x) - f''(y)\| \leq M \|x - y\|$
\item<3-> начальная точка $x_0$ достаточно близка к $x^*$: $\|x_0 - x^*\| \leq \frac{2\mu}{3M}$
\end{itemize}
\uncover<4->{тогда метод Ньютона сходится квадратично
\[
\|x_{k+1} - x^* \| \leq \frac{M\|x_k - x^*\|^2}{2(\mu - M\|x_k - x^*\|)} 
\]
}
\end{Th}
\end{frame}

\begin{frame}{Пример сходимости}

\end{frame}

\begin{frame}{Доказательство}

\end{frame}

\begin{frame}{Pro \& Contra}

\end{frame}




\begin{frame}{Квазиньютоновские методы}

\end{frame}

\begin{frame}{Метод Barzilai-Borwein}

\end{frame}

\begin{frame}{Метод DFP}

\end{frame}

\begin{frame}{Метод BFGS}

\end{frame}

\begin{frame}{Квазиньютоновские методы с ограниченной памятью}

\end{frame}

\begin{frame}{Метод L-BFGS}

\end{frame}

\begin{frame}{Pro \& Contra}

\end{frame}


\end{document}